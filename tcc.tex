\documentclass{ifbtcc}
% ----
% Início do documento
% ----
\begin{document}

% Seleciona o idioma do documento (conforme pacotes do babel)
%\selectlanguage{english}
\selectlanguage{brazil}

% Retira espaço extra obsoleto entre as frases.
\frenchspacing 

% ----------------------------------------------------------
% ELEMENTOS PRÉ-TEXTUAIS
% ----------------------------------------------------------
% \pretextual

% ---
% Capa
% ---
\imprimircapa
% ---

% ---
% Folha de rosto
% (o * indica que haverá a ficha bibliográfica)
% ---
\imprimirfolhaderosto*
% ---

% ---
% Inserir a ficha bibliografica
% ---

% Isto é um exemplo de Ficha Catalográfica, ou ``Dados internacionais de
% catalogação-na-publicação''. Você pode utilizar este modelo como referência. 
% Porém, provavelmente a biblioteca da sua universidade lhe fornecerá um PDF
% com a ficha catalográfica definitiva após a defesa do trabalho. Quando estiver
% com o documento, salve-o como PDF no diretório do seu projeto e substitua todo
% o conteúdo de implementação deste arquivo pelo comando abaixo:
%
% \begin{fichacatalografica}
%     \includepdf{fig_ficha_catalografica.pdf}
% \end{fichacatalografica}

\begin{fichacatalografica}
	\sffamily
	\vspace*{\fill}					% Posição vertical
	\begin{center}					% Minipage Centralizado
	\fbox{\begin{minipage}[c][8cm]{13.5cm}		% Largura
	\small
	\imprimirautor
	%Sobrenome, Nome do autor
	
	\hspace{0.5cm} \imprimirtitulo  / \imprimirautor. --
	\imprimirlocal, \imprimirdata-
	
	\hspace{0.5cm} \thelastpage p. : il. (algumas color.) ; 30 cm.\\
	
	\hspace{0.5cm} \imprimirorientadorRotulo~\imprimirorientador\\
	
	\hspace{0.5cm}
	\parbox[t]{\textwidth}{\imprimirtipotrabalho~--~\imprimirinstituicao,
	\imprimirdata.}\\
	
	\hspace{0.5cm}
		1. Palavra-chave1.
		2. Palavra-chave2.
		2. Palavra-chave3.
		I. Orientador.
		II. Universidade xxx.
		III. Faculdade de xxx.
		IV. Título 			
	\end{minipage}}
	\end{center}
\end{fichacatalografica}
% ---

% ---
% Inserir errata
% ---
\input{elementos-pre-textuais/errata.tex}
% ---

% ---
% Inserir folha de aprovação
% ---

\input{elementos-pre-textuais/folha-de-aprovacao.tex}

% ---

% ---
% Dedicatória
% ---
\input{elementos-pre-textuais/dedicatoria.tex}
% ---

% ---
% Agradecimentos
% ---
\input{elementos-pre-textuais/agradecimentos.tex}
% ---

% ---
% Epígrafe
% ---
\input{elementos-pre-textuais/epigrafe.tex}
% ---

% ---
% RESUMOS
% ---

% resumo em português
\input{elementos-pre-textuais/resumo.tex}

% resumo em inglês
\input{elementos-pre-textuais/abstract.tex}
% ---

% ---
% inserir lista de ilustrações
% ---
\input{elementos-pre-textuais/lista-de-ilustracoes.tex}
% ---

% ---
% inserir lista de quadros
% ---
\input{elementos-pre-textuais/lista-de-quadros.tex}
% ---

% ---
% inserir lista de tabelas
% ---
\input{elementos-pre-textuais/lista-de-tabelas.tex}
% ---

% ---
% inserir lista de abreviaturas e siglas
% ---
\input{elementos-pre-textuais/lista-de-abreviaturas.tex}
% ---

% ---
% inserir lista de símbolos
% ---
\input{elementos-pre-textuais/lista-de-simbolos.tex}
% ---

% ---
% inserir o sumario
% ---
\input{elementos-pre-textuais/sumario.tex}
% ---



% ----------------------------------------------------------
% ELEMENTOS TEXTUAIS
% ----------------------------------------------------------
\textual

% ----------------------------------------------------------
% Introdução (exemplo de capítulo sem numeração, mas presente no Sumário)
% ----------------------------------------------------------
\input{elementos-textuais/introducao.tex}

% ---
% Capitulo de revisão de literatura
% ---
\chapter{Lorem ipsum dolor sit amet}
% ---

% ---
\section{Aliquam vestibulum fringilla lorem}
% ---

\lipsum[1]

\lipsum[2-3]

% ----------------------------------------------------------
% PARTE
% ----------------------------------------------------------
\part{Resultados}
% ----------------------------------------------------------


% ---
% primeiro capitulo de Resultados
% ---
\input{elementos-textuais/resultados.tex}

% ---
% Conclusão
% ---
\input{elementos-textuais/conclusao.tex}
% ----------------------------------------------------------
% ELEMENTOS PÓS-TEXTUAIS
% ----------------------------------------------------------
\postextual
% ----------------------------------------------------------

% ----------------------------------------------------------
% Referências bibliográficas
% ----------------------------------------------------------
\bibliography{referencias/referecias}

% ----------------------------------------------------------
% Glossário
% ----------------------------------------------------------
%
% Consulte o manual da classe abntex2 para orientações sobre o glossário.
%
%\glossary

% ----------------------------------------------------------
% Apêndices
% ----------------------------------------------------------

% ---
% Inicia os apêndices
% ---

\begin{apendicesenv}

% Imprime uma página indicando o início dos apêndices
\partapendices
\input{elementos-pos-textuais/apendice.tex}
\end{apendicesenv}
% ---


% ----------------------------------------------------------
% Anexos
% ----------------------------------------------------------

% ---
% Inicia os anexos
% ---
\begin{anexosenv}

% Imprime uma página indicando o início dos anexos
\partanexos

\input{elementos-pos-textuais/anexo.tex}

\end{anexosenv}

%---------------------------------------------------------------------
% INDICE REMISSIVO
%---------------------------------------------------------------------
\input{elementos-pos-textuais/indice-remissivo.tex}
%---------------------------------------------------------------------

\end{document}